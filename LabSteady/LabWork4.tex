% !TeX program = lualatex
% !TeX encoding = utf8
% !TeX spellcheck = uk_UA
% !BIB program = bibtex8

\documentclass{LabWork}
\usetikzlibrary{arrows.meta}
\tikzset{
every info/.style={font=\small},
}

%============================================= Заголовок документу ====================================================%
\work{4}
\title{Джерела електричної енергії та їх характеристики}

\author{Тор А.~В.}{}
\author{Другий А.~В.}{}

\group{ФФ-93}

\abstract{%

Виміряти напругу $U$, на клемах декількох джерел електроенергії як функцію сили струму, змінюючи зовнішній опір $R_e$, і розрахувати напругу без навантаження $\mathcal{E}$ і внутрішній опір $R_i$ джерела.

Виміряти безпосередньо напругу без навантаження від джерела електроенергії (без зовнішнього опору) та його внутрішній опір (підбираючи умови узгодження $R_i  =  R_e$). Визначити силову діаграму залежності між напругою на клемах та силою струму.
}
%======================================================================================================================%

\begin{document}
\writedatatofile{\jobname}
\maketitle

\section{Теоретичні відомості}

\begin{wrapfigure}{l}{0.4\linewidth}\centering
	\begin{tikzpicture}[thick, every circuit symbol/.style={thick},large circuit symbols,]
		\draw (0,-2) coordinate (START) to [battery={info={$\mathcal{E}$}}] ++(0,4) -- ++(4,0) to [resistor={info=$R_e$}] ++(0,-4) -- (START);
		\node[contact] (1)  at (2.5,-2) {}; \node[below] at (1) {$1$};
		\node[contact] (2)  at (2.5,+2) {}; \node[above] at (2) {$2$};
		\draw[-latex, ultra thick, blue, font=\small] (1) -- node[left] {$U_{12}$} (2);
		\draw[-latex, ultra thick, red, font=\small, yshift=10pt] (0.5,-2) -- node[above] {$I$} ++(1,0);
	\end{tikzpicture}
	\caption{Електричне коло}
	\label{thpic1}
\end{wrapfigure}
Електричним струмом називають упорядкований рух електричних зарядів в провідному середовищі під дією електричного поля.

Мірою інтенсивності руху електричних зарядів в провідниках є величина струму або просто струм $I$.
Величина струму~--- це кількість електричних зарядів, що протікають через поперечний переріз провідника за одиницю часу:
\begin{equation}\label{Idef}
	I = \frac{dq}{dt}.
\end{equation}

Якщо струм в часі не змінюється за величиною та напрямком ($I = \mathrm{const}$) то такий струм називають постійним, тому для постійного струму можемо написати:
\begin{equation}\label{Idef1}
	I = \frac{q}{t}.
\end{equation}

Електричний струм на ділянці виникає в тому випадку, якщо на його затискачах (полюсах) створена різниця потенціалів (існує електричне поле вздовж ділянки). Різниця потенціалів між двома точками ділянки називають напругою або падінням напруги. Потенціал заданої точки позначають як $\phi$ з відповідним індексом, наприклад для точок $1$ і $2$ потенціали $\phi_1$ та $\phi_2$, відповідно. напруга між двома точками $U_{12} = \phi_1 - \phi_2$ (рис.~\ref{thpic1}).

Такий запис означає, що $\phi_1 > \phi_2$, а за додатній напрямок струму для будь-якої ділянки кола прийнято напрямок від точки вищого потенціалу до нижчого. Таким чином, на ділянці кола $1 -2$ напрямок струму $I$ збігається з напрямком напруги $U_{12}$. Електрична напруга чисельно дорівнює роботі $A$, яку здійснює джерело електричної енергії (або джерело живлення) при переміщенні одиничного заряду $q$ з однієї точки в іншу, наприклад, з точки $1$ в точку $2$:
\begin{equation}\label{Udef}
	U_{12} = \frac{A_{12}}{q}.
\end{equation}

Потужність~--- робота, яка виконується джерелом одиницю часу $t$,
що розвивається на цій ділянці:
\begin{equation}\label{Pdil}
	P = \frac{A_{12}}{t} = \frac{U_{12} q}{t} = U_{12}I.
\end{equation}

Для переміщення заряду по замкнутому колу джерело електроенергії повинно виконати роботу, що
чисельно дорівнює ЕРС $\mathcal{E} = \frac{A}{q}$. У цьому випадку потужність,
що розвивається джерелом дорівнює:
\begin{equation}\label{Psource}
	P = \mathcal{E}I.
\end{equation}

Електричні властивості джерела електричної енергії характеризуються його \emph{внутрішнім опором}. Під внутрішнім опором генератора розуміють опір електричному струму всіх елементів всередині генератора.

\begin{wrapfigure}{l}{0.4\linewidth}\centering
	\begin{tikzpicture}[thick, every circuit symbol/.style={thick},large circuit symbols,]
		\draw (0,0) node [contact] {} to [battery={info={$\mathcal{E}$}}] ++(2,0) to [resistor={info=$R_i$, color=red}] ++(2,0) node [contact] {} ;
	\end{tikzpicture}
	\caption{Еквівалентна схема джерела}
	\label{thpic2}
\end{wrapfigure}
Врахування внутрішнього опору джерела необхідне при аналізі і розрахунку режимів електричного кола. Разом з тим при розрахунку електричних кіл внутрішній опір $R_i$ джерела може виявитися у багато разів менше опору зовнішнього кола $R_e$. У цих випадках внутрішній опір джерела можна прийняти рівним нулю, що дозволить вважати напругу на затискачах джерела не залежною від струму навантаження і рівним ЕРС джерела. Джерело з внутрішнім опором, рівним нулю, називають джерелом напруги. Якщо внутрішній опір генератора $R_i > 0$, то таке джерело зображують у вигляді джерела ЕРС і послідовно з'єднаного з ним елемента $R_i$ (рис.~\ref{thpic2}). Надалі будемо вважати, що ЕРС джерела і його внутрішній опір не залежать від струму в колі.

Якщо внутрішній опір джерела дуже великий, струм у зовнішньому колі практично не залежить від опору самого кола. У цих випадках джерело характеризується не ЕРС, а струмом і називається джерелом струму.

В електричному колі (рис.~\ref{thpic1}) при зміні опору навантаження $R_e$ від нуля до нескінченності змінюється струм $I$ і напруга $U$. Розглянемо найбільш характерні режими роботи джерела електричної енергії при зміні опору навантаження $R_e$.

Таких режимів чотири:
\begin{enumerate}
	\item \emph{режим холостого ходу},
	\item \emph{номінальний},
	\item \emph{режим короткого замикання},
	\item \emph{режим узгодження}.
\end{enumerate}


У режимі холостого ходу опір навантаження $R_e \to \infty$. Струм в колі дорівнює нулю, а напруга на затискачах джерела енергії найбільша і дорівнює ЕРС: $U = \mathcal{E}$. У номінальному режимі від джерела електроенергії відбирається номінальна потужність, тобто та найбільша потужність, яку може довго розвивати джерело не перегріваючись. Нагрівання джерела визначається потужністю втрат  в ньому, яка пропорційна значенню внутрішнього опору і квадрату струму: $\Delta P = I^2R_e$. Поняття номінального режиму відноситься також і до приймача, який при перевантаженні може нагрітися до неприпустимо високої температури.

Для кожного джерела і приймача номінальний режим передбачає цілком певну номінальне навантаження, відповідне номінальним значенням струму і напруги.

%---------------------------------------------------------
\begin{wrapfigure}{l}{0.45\linewidth}%[h!]
	\centering
	\begin{tikzpicture}[
			mark position/.style args={#1(#2)}{
					postaction={
							decorate,
							decoration={
									markings,
									mark=at position #1 with \coordinate (#2);
								}
						}
				}
		]
		\draw[-latex, thick] (0,0) -- ++(6.5,0) node[below] {$I$};
		\draw[-latex, thick] (0,0) -- ++(0,4) node[left] {$\eta$};
		\draw[-latex, thick] (6,0) node[below] {$I_0$}-- ++(0,7) node[right] {$P$};
		\node[below] at (3,0) {$I_0/2$};
		\node[left] at (0,3) {$1$};
		\draw[dashed] (3,0) -- (3,3/2);
		\draw[dashed] (0,3/2) node[left] {$0.5$} -- (3,3/2);
		\draw[ultra thick, domain=0:6, smooth, variable=\x, red,  mark position=0.8(A)]  plot ({\x}, {-1/6*(\x)*(\x-6)}) ;\node[above] at (A) {$P_e$};

		\draw[ultra thick, domain=0:6, smooth, variable=\x, blue,  mark position=0.2(E)] plot ({\x}, {-1/2*\x+3}); \node[above] at (E) {$\eta$};

		\draw[ultra thick, domain=0:6, smooth, variable=\x, green!60!black,  mark position=0.7(Pe)] plot ({\x}, {1/6*\x*\x}); \node[anchor=north west] at (Pe) {$I^2R_i$};

		\draw[ultra thick, domain=0:6, smooth, variable=\x, green!60!black,  mark position=0.7(P)] plot ({\x}, {1*\x});\node[anchor=south east] at (P) {$P$};
	\end{tikzpicture}
	\caption{Діаграма режимів роботи джерела електроенергії}
	\label{thpic3}
\end{wrapfigure}
%---------------------------------------------------------
Важливим показником раціональної роботи джерела електричної енергії є коефіцієнт корисної дії (ККД) $\eta$. Він визначається відношенням потужності в навантаженні ($P_e = I^2R_e$) до повної потужності, що виробляється (генерується) джерелом електроенергії ($P = \mathcal{E}I$):
\begin{equation}\label{eta}
	\eta = \frac{P_e}{P} = \frac{I^2R_e}{\left( I^2R_i + I^2R_e\right) } = \frac{R_e}{R_i + R_e}.
\end{equation}

У режимі короткого замикання, коли $R_e = 0 $, струм в колі буде обмежений тільки внутрішнім опором джерела електроенергії $I_0 = \frac{\mathcal{E}}{R_i}$, $P_e = 0$ і $P = \mathcal{E} I_0$. У цьому випадку $\eta = 0$. Для джерела з малим внутрішнім опором (акумулятори, електромашинні генератори) режим короткого замикання небезпечний~--- це аварійний режим. Для гальванічних елементів режим короткого замикання менш небезпечний, так як їх внутрішній опір відносно великий.

При узгодженому режимі в приймачі (навантаженні) виділяється найбільша потужність. Такий режим використовується в вимірювальних колах, в пристроях обчислювальної, інформаційної техніки, засобів зв'язку. В цьому випадку потужність в навантаженні $P_e$ дорівнює потужності джерела $P$ за
винятком внутрішніх втрат в самому джерелі, тобто $P_e = P - I^2R_i$, а тому коефіцієнт корисної дії $\eta = 1 - \frac{I}{I_0}$, де $I_0$~--- струм короткого замикання. Неважко зрозуміти, що в режимі узгодження потужність в приймачі (навантаженні) буде максимальною і  дорівнюватиме половині потужності джерела:
\begin{equation}\label{key}
	P_e = \frac{\mathcal{E}^2}{4R_i},
\end{equation}
а ККД при цьому буде дорівнювати $50$~\%.

При передачі великих потужностей робота в узгодженому режимі, як правило, є неприпустимою. У колах великої потужності неодмінною умовою є $R_e \gg R_i$, тобто забезпечення великих ККД.

На рис.~\ref{thpic3} побудовані залежності: $\eta$, $P_e$, $I^2R_i$ та $P$. З наведених кривих випливає, що найбільша потужність в навантаженні буде при $I =I_0 / 2$ або, що те ж, при $R_e = R_i$.

\section{Теоретичне підґрунтя роботи}

%---------------------------------------------------------
\begin{wrapfigure}{l}{0.5\linewidth}\centering
	\begin{tikzpicture}[thick, every circuit symbol/.style={thick},large circuit symbols,]
		%			\draw[blue, fill=blue!20] (1,1.5) rectangle ++(2,2);
		\draw[red, fill=red!20] (-1.2,-1.7) rectangle ++(2.2,3.1);
		\draw (0,-2) coordinate (START) to [resistor={info={$R_i$}, color=red}] ++(0,2) to [battery={info={$\mathcal{E}$}}] ++(0,2) to [ampermeter] coordinate (A1) ++(6,0) to [resistor={info'=$R_e$}] ++(0,-4) -- (START)
		;
		\draw [latex-] (6.1,0) -- ++(0.5,0) -- ++(0,-1) -- ++(-0.6,0) node[contact] {};
		\draw ([xshift=-1cm]A1) node[contact] {} to[voltmeter] ++(0,-4) node[contact] {};
	\end{tikzpicture}
	\caption{Послідовне з'єднання елементів}
	\label{pic1}
\end{wrapfigure}
%---------------------------------------------------------

Напруга на клемах джерела електроенергії та сила струму залежать від навантаження, тобто від зовнішнього опору $R_e$. Реальні джерела можуть бути описані в еквівалентній схемі як послідовно з'єднані ідеальне \emph{джерело напруги} $\mathcal{E}$ і внутрішній опір $R_i$ (рис~\ref{pic1}).


Якщо джерело електроенергії підключене до зовнішнього опору $R_e$, то згідно закону Ома буде протікати струм:
\begin{equation}\label{OhmLow}
	I = \frac{\mathcal{E}}{R_i + R_e}.
\end{equation}

Вихідна напруга (напруга на клемах джерела) дорівнює:
\begin{equation}\label{U}
	U = \mathcal{E} - R_i I.
\end{equation}

В цій роботі напруга на клемах визначається як функція сили струму. B ідеалі існує лінійна залежність між напругою $U$, та силою струму $I$. Зов\-ніш\-ній опір $R_e$ визначає відношення напруги до струму в робочій точці:
\begin{equation}\label{Re}
	R_e = \frac{U}{I}.
\end{equation}

На діаграмі (рис~\ref{pic2}) йому відповідає тангенс куту нахилу прямої, шо проведена з початку координат до робочої точки. Як це витікає з \eqref{U}, тангенс кута нахилу самої залежності $U(I)$ дорівнює:
\begin{equation}\label{tan}
	\tg\varphi = - R_i.
\end{equation}

%---------------------------------------------------------
\begin{figure}[h!]\centering
	\begin{tikzpicture}
		\draw[-latex] (0,0) -- ++(5,0) node[below] {$I$, А};
		\draw[-latex] (0,0) -- ++(0,5) node[left] {$U$, В};
		\draw[red, thick] (0,4) -- (4,0) coordinate[pos=0.5] (X) coordinate[pos=0.4] (A);
		\draw[dashed] (X) -- (0,0-|X) node[below] {$I_0/2$};
		\draw[dashed] (X) -- (0,0|-X) node[left] {$\mathcal{E}/2$};
		\fill (X) circle (0.1);
		\node[right] at (X) {$R_e = R_i$};
		\node[left] at (0,4) {$\mathcal{E}$};
		\node[below] at (4,0) {$I_0$};
		\node[right] at (0,4) {$R_e \to \infty$};\node[above right] at (4,0) {$R_e \to 0$};
		\draw (3,0) arc(180:{180-45}:1) node[anchor=north east, xshift=-5pt, font=\small] {$\pi - \varphi$};
	\end{tikzpicture}
	\caption{Характеристика джерела із сталим внутрішнім опором $R_i$.}
	\label{pic2}
\end{figure}
%---------------------------------------------------------

Таким чином, лінійна залежність означає, шо внутрішній опір $R_e$, є сталим.

Для нашої роботи, ми будемо аналізувати три типи навантаження:

\begin{enumerate}
	\item Холостий хід (або узгодження напруг) $R_e \to \infty$. Струм не тече, тому  немає падіння напруги на опорі $R_i$, при цьому $U = \mathcal{E}$.
	\item Коротке замикання (або узгодження струмів) $R_e=0$. Падіння напруги на внутрішньому опорі становить $\mathcal{E}$, тому $U = 0$. Струм, який тече при короткому замиканні, визначається як
	      \begin{equation}\label{short}
		      I_0=\frac{\mathcal{E}}{R_i}.
	      \end{equation}
	\item Узгодження потужності (або узгодження опорів)  $R_e=R_i$. В цьому випадку $U = \frac{\mathcal{E}}{2}$, $I=\frac{I_0}{2}$.
\end{enumerate}

Пронормуємо виміряні значення напруги $U$, та струму $I$ на максимальні значення для джерела електроенергії, тобто напругу холостого ходу $\mathcal{E}$ та струм короткого замикання $I_0$, відповідно. З рівнянь \eqref{OhmLow} та \eqref{short} отримуємо:
\begin{equation}\label{I/I0}
	\frac{I}{I_0} = \frac{1}{1 + \frac{R_e}{R_i}}.
\end{equation}

Рівняння \eqref{U} дає
\begin{equation}\label{Г/Г0}
	\frac{U}{\mathcal{E}} = 1 - \frac{1}{1 + \frac{R_e}{R_i}},
\end{equation}
або враховуючи \eqref{I/I0}, матимемо:
\begin{equation}
	\frac{U}{\mathcal{E}} + \frac{I}{I_0}  = 1.
\end{equation}

Тобто нормовані залежності напруги та струму від зовнішнього навантаження суть взаємно-протилежні (рис~\ref{pic3}).

Потужність, що розсіюється на резисторі $R_e$, дорівнює:
\begin{equation}\label{P}
	P = I^2R_e.
\end{equation}

Її доречно нормалізувати на потужність, шо розсіюється на опорі $R_e$, у випадку короткого замикання:
\begin{equation}\label{Pkz}
	P_0 = \frac{\mathcal{E}^2}{R_i},
\end{equation}

Тоді
\begin{equation}\label{P/P0}
	\frac{P}{P_0} = \frac{U}{\mathcal{E}} \cdot \frac{I}{I_0} = \frac{R_e/R_i}{\left( 1+ R_e/R_i \right)^2 }.
\end{equation}

Залежність нормованої вихідної потужності $Р/Р_0$ від навантаження має максимум за умови $R_e=R_i$ (рис~\ref{pic3}). Тобто джерело видає в навантаження максимальну потужність у випадку узгодження потужностей.

%---------------------------------------------------------
\begin{figure}[h!]\centering
	\begin{tikzpicture}
		\begin{semilogxaxis}[
				x label style={at={(axis description cs:0.5,-0.1)},anchor=north},
				legend style={row sep = 5pt},
				xlabel={$\frac{R_e}{R_i}$},
				width = \linewidth,
				height = 0.75\linewidth,
				grid = both,
				major grid style={line width=.6pt,draw=brown!60},
				minor tick num = 9,
				minor grid style = {line width=.1pt,draw=brown!20},
				axis lines=center,
				%        axis y line shift=1,
			]
			\addplot[domain=0.01:100, samples=100, blue, thick] {1 - 1/(1 + x/1)};
			\addplot[domain=0.01:100, samples=100, green!50! black, thick] {1/(1 + x/1)};
			\addplot[domain=0.01:100, samples=100, red, thick] {x/1/(1+x/1)^2};
			\node (XX) [draw=red, fill= white, minimum width=5cm] at (axis cs:10,0.5) {Холостий хід};
			\node (KZ) [draw=red, fill= white, minimum width=5cm] at (axis cs:0.1,0.5) {Коротке замикання};
			\draw[-latex, double] (XX.east) -- ++(0.5,0);
			\draw[-latex, double] (KZ.west) -- ++(-0.5,0) ;
			\legend{${U}/{\mathcal{E}}$, ${I}/{I_0}$, ${P}/{P_0}$}
		\end{semilogxaxis}
	\end{tikzpicture}
	\caption{Діаграма режимів для джерела електроенергії}
	\label{pic3}
\end{figure}
%---------------------------------------------------------

\section{Експериментальне устаткування}

Експериментальне устаткування складається з кислотно-лужного акумулятора, змінного навантаження, в якості якого використовується реостат на $100$~Ом або реостат на $10$~Ом; вимірювальні прилади --- високоомного цифрового мультиметра, який використовується в якості вольтметра, та звичайного магнітоелектричного мультиметра (клас точності $0.5$), який використовується в якості амперметра (клас точності $0.5$).

\section{Хід експерименту}

\begin{enumerate}
	\item Збираємо вимірювальну схему згідно (рис.~\ref{pic1}). В якості змінного резистора $R_e$ використовуємо 100-омний реостат. Для зручності змінюємо струм $I$ з кроком $0.1$~А. Вихідну напругу вимірюємо високоомним цифровим вольтметром.
	\item Знімаємо залежність $U(I)$ для наявних джерел електроенергії.
	\item Визначаємо внутрішній опір та напругу холостого ходу $\mathcal{E}$  з лінійної апроксимації залежності $U(I)$.
	\item Побудуйте нормовані діаграми, аналогічні (рис.~\ref{pic3}).
\end{enumerate}



\section*{Контрольні питання}

\begin{enumerate}
	\item Чому цифровий мультиметр використовують в якості вольтметра, а звичайний --- в якості амперметра, а не навпаки?
	\item Які відмінності між приладами джерелами струму та напруги?
	\item Який принцип дії гальванічних елементів?
	\item Який принцип дії акумулятора?
	\item Чим відрізняється реостатне та потенціометричне підключення?
	\item Метод зарядки акумулятора. Чи можна таким чином зарядити гальванічний елемент?
	\item Напрямок ліній електричного поля поблизу від проволоки зі струмом.
	\item Що таке електричний струм та густина струму?
	\item Від чого залежить потужність джерела електроенергії?
	\item Назвати одиниці виміру струму, напруги, потужності в СГС та SI.
	\item Які ви знаєте методи виміру струму, напруги, потужності?
\end{enumerate}
\end{document}
