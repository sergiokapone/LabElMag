% !TeX program = lualatex
% !TeX encoding = utf8
% !TeX spellcheck = uk_UA

\documentclass{LabWork}
\graphicspath{{LabWork7pic/}}
%============================================= Заголовок документу ====================================================%
\work{7}
\title{Магнітне поле Землі}

\author{Тор А.~В.}{}
\author{Другий А.~В.}{}

\group{ФФ-93}

\abstract{Визначити момент сили, зумовлений магнітним моментом в постійному магнітному полі, як функцію:
\begin{itemize}
    \item індукції магнітного поля;
    \item кута між напрямком магнітного поля та магнітного моменту;
    \item величини магнітного моменту.
\end{itemize}
}
%======================================================================================================================%

\begin{document}
\writedatatofile{\jobname}
\maketitle


\section{Теоретичне підґрунтя}
\subsection{Земний магнетизм}

Про існування магнетизму було відомо з глибокої давнини. Вважається, що перший компас з'явився в Китаї. У 1600 році в книзі  <<Про магніті, магнітних тілах і про великий магніт~--- Землю>> У.~Гільбертом  було дано уявлення про причини земного магнетизму. У 1785 почалися розробки способу вимірювання напруженості магнітного поля, що базується на методі крутного моменту, запропонованому Ш. Кулоном. У 1839 К. Гаусс теоретично обгрунтував метод вимірювання горизонтальної складової вектора магнітного поля планети. На початку XX ст. було визначено зв'язок між магнітним полем Землі і її будовою.

Вже з середини XX століття загальноприйнято, що походження геомагнітного поля і основні чинники його еволюції пов'язані з процесами в рідкому зовнішньому ядрі Землі.

Причина подібної впевненості полягає в наступному. Сукупність спостережних даних про магнітне поле Землі переконує нас в тому, що воно має планетарний характер і його джерела повинні знаходитися глибоко під поверхнею Землі. Спроба зв'язати таке магнітне поле з величезним постійним магнітом входить в протиріччя з помітним зростанням температури вглиб Землі. Справді, феромагнітні властивості зникають при досягненні критичної температури~--- званої точкою Кюрі, та й сам образ гігантського постійного магніту десь в глибині Землі чи не здається реалістичним.

Ще одним джерелом магнітного поля Землі в принципі могло б служити поділ зарядів, в результаті якого область між земною поверхнею і іоносферою являє собою гігантський конденсатор. Обертання Землі призводить до руху заряду цього конденсатора і виникає таким чином електричний струм створює магнітне поле. Однак оцінки показують, що його напруженість набагато нижче ніж та, що спостерігається.

Ще одним джерелом магнітного поля Землі може бути пов'язане з явищем електромагнітної індукції Фарадея. Цей механізм генерації магнітного поля називається механізмом динамо. Суть механізму динамо зазвичай пояснюють як перетворення кінетичної енергії рухів електропровідної рідини (плазми) в магнітну енергію.

Вперше механізм динамо був запропонований на початку XX століття для пояснення походження магнітного поля Сонця. Пізніше з дією цього механізму стали пов'язувати походження магнітних полів майже всіх небесних тіл, що мають магнітне поле. Відзначимо, що для більшості небесних тіл питання про походження магнітних полів стоїть ще гостріше, ніж для Землі: тверді тіла в космосі~--- велика рідкість, а уявлення про газоподібний феромагнетик вже зовсім ні з чим не в'яжеться.

Пояснити походження геомагнітного поля механізмом динамо теж непросто. Справа в відомому правилі Ленца, згідно з яким додатковий струм, який з'являється в рамці зі струмом, що рухається в початковому магнітному полі, спрямований так, щоб зменшити початкове магнітне поле. Іншими словами, найпростіші потоки провідної рідини, включаючи диференціальне обертання, не можуть призводити до самозбудження магнітного поля. Для того, щоб обійти правило Ленца, потрібні дві рамки з струмом, які взаємно підсилюють магнітні поля, що пронизують ці рамки.

Вказати конкретний реалістичний механізм, який призводить до самозбудження магнітного поля в результаті дії електромагнітної індукції, вдалося лише в 50-60 рр. XX ст. Відповідь виявилася настільки несподіваною, що вона зажадала багаторічних зусиль трьох наукових груп.

Суть ідеї полягає в наступному. Магнітне поле у вакуумі, що створюється електричним струмом, перпендикулярно цьому струму. Однак виявилося, що магнітне поле в випадковому, турбулентному або конвективному потоці, усереднене по пульсаціям цього потоку, набуває компоненти, паралельної електричному струму.

\section{Хід роботи}



\section{Завдання}


\section{Контрольні запитання}



\section{Розрахункові завдання}




\section{Результи вимірювань та обробка експериментальних даних}


\section{Обговорення результатів}


\section*{Висновки}

\end{document}
