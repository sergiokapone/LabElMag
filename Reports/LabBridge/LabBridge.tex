% !TeX program = lualatex
% !TeX encoding = utf8
% !TeX spellcheck = uk_UA
% !BIB program = bibtex8

\documentclass{LabWork}
\graphicspath{{LabWork5pic/}}
\usetikzlibrary{arrows.meta}
\tikzset{
every info/.style={font=\small},
}
\usetikzlibrary{decorations.pathreplacing}

%============================================= Заголовок документу ====================================================%
\work{5}
\title{Вимірювання величини електричного опору за допомогою містка Уітстона}

%\author{}{}

%\group{ФФ-93}

\abstract{%

Вивчення принципу роботи вимірювальної мостової схеми. Визначення величини опору двох провідників і величини опору при їх послідовному і паралельному з'єднанні. Визначення величини внутрішнього опору гальванометра.
}
%======================================================================================================================%

\begin{document}
\writedatatofile{\jobname}
\maketitle

\section{Таблиці та розрахунки}

\pgfplotstableread{
R l1  l2 
10  0.920 0.080
50  0.668 0.332
100 0.502 0.498
}\ExpData

% ===================== Додаємо розрахунок R_x по формулі (4)
\pgfplotstablecreatecol[
	create col/expr={\thisrow{R}*\thisrow{l1}/\thisrow{l2}},
]{Rx}\ExpData



%================ Верстка таблиці ======================
\begin{table}[!htbp]\centering
	\caption{Таблиця з експериментальними даними та розрахованим значенням $R_x$}
	\label{tab1}
	\pgfplotstabletypeset[
		multicolumn names,
		columns/R/.style={
				column name={$R$, Ом},
				column type={
						S[
								round-mode=places,
								round-precision=2,
								%                scientific-notation=engineering,
								%                table-format=2.1,
								exponent-product = \cdot
							]
					},
				string type
			},
		columns/l1/.style={
				column name={$l_1$, м},
				column type={
						S[
								round-mode=places,
								round-precision=2,
								%                scientific-notation=engineering,
								%                table-format=2.1,
								exponent-product = \cdot
							]
					},
				string type
			},
		columns/l2/.style={
				column name={$l_2$, м},
				column type={
						S[
								round-mode=places,
								round-precision=2,
								%                scientific-notation=engineering,
								%                table-format=2.1,
								exponent-product = \cdot
							]
					},
				string type
			},
		columns/Rx/.style={
				column name={$R_x$, Ом},
				column type={
						S[
								round-mode=places,
								round-precision=2,
								%                scientific-notation=engineering,
								%                table-format=2.1,
								exponent-product = \cdot
							]
					},
				string type
			},
		every head row/.style={
				before row={\toprule},
				after row={\midrule},
			},
		every last row/.style={after row=\bottomrule}
	]\ExpData
\end{table}

\averageVal{\ExpData}{Rx}{\averageRx}% --- обчислення середнього з колонки Rx і присвоєння його змінній \averageRx
\maxVal{\ExpData}{Rx}{\maxRx}% --- обчислення максимального з колонки Rx і присвоєння його змінній \maxRx
\minVal{\ExpData}{Rx}{\minRx}% --- обчислення мінімального з колонки Rx і присвоєння його змінній \maxRx

Середнє значення $R_x$ дорівнює \pgfmathprintnumber{\averageRx}~Ом. % Верстка значення

Максимальне значення $R_x$ в колонці дорівнює \pgfmathprintnumber{\maxRx}~Ом. % Верстка значення

Мінімальне значення $R_x$ в колонці дорівнює \pgfmathprintnumber{\minRx}~Ом. % Верстка значенн


\end{document}
