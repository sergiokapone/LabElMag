% !TeX program = lualatex
% !TeX encoding = utf8
% !TeX spellcheck = uk_UA

\documentclass{LabWork}
\graphicspath{{LabWork6pic/}}
%============================================= Заголовок документу ====================================================%
\work{6}
\title{Магнітний момент у магнітному полі}

%\author{Тор А.~В.}{}
%\author{Другий А.~В.}{}
%
%\group{ФФ-93}

\abstract{Визначити момент сили, зумовлений магнітним моментом в постійному магнітному полі, як функцію:
\begin{itemize}
\item індукції магнітного поля;
\item кута між напрямком магнітного поля та магнітного моменту;
\item величини магнітного моменту.
\end{itemize}
}
%======================================================================================================================%

\begin{document}
\maketitle

%
\fputrue% ----- Увімкнення двигуна точних розрахунків
\pgfmathsetmacro{\muz}{4*pi*1e-7}
\pgfmathsetmacro{\R}{0.2}
\pgfmathsetmacro{\n}{154}
\pgfmathsetmacro{\CG}{(4/5)^(3/2)*\muz*\n/\R}
\fpufalse% ----- Вимкнення двигуна точних розрахунків
%
\section{Робоча формула}

Якщо магнітне поле буде неоднорідним, то воно буде різне на різних частини  контура, який вміщений у це магнітне поле і, відповідно, на різні частини контура діяти різний обертальний момент.Для того, щоб цього уникнути, бажано використовувати однорідне магнітне поле, що забезпечується завдяки котушкам Гельмгольца.

У даній роботі контур~--- це пласка петля, що має $N$ витків, по якій тече постійний струм $I'$, діаметр кільця $d$. А тому, його магнітний момент дорівнює:
\begin{equation*}\label{eq:pmE}
	p_m = I'\frac{N \pi d^2}{4},
\end{equation*}
а обертальний момент тоді визначатиметься як:
\begin{equation*}\label{}
	M = I'\frac{N \pi d^2}{4} B\sin\alpha.
\end{equation*}

Далі, враховуючи, що магнітне поле котушок Гельмгольца пропорційні силі струму, що тебе по ним ($B = CI$, де $C$~--- величина, що залежить від параметрів котушок і називається \emph{константою котушок Гельмгольца}) можемо записати:
\begin{equation}\label{eq:TorqueE}
	M = \frac{\pi}{4} CII'Nd^2\sin\alpha.
\end{equation}


\subsection{Дослід 1. Залежність обертального моменту від сили струму в \\контурі}

Константами експерименту є величини в табл.~\ref{tab:ExpConst}.

\pgfmathsetmacro{\N}{3}
\pgfmathsetmacro{\D}{0.12}
\pgfmathsetmacro{\angle}{90}
\pgfmathsetmacro{\IG}{2.85}

\begin{table}[h!]
	\centering
	\caption{Константи експерименту}
	\begin{tabular}{ll}
		\toprule
		Величина                                 & Значення                        \\ \midrule
		Число витків контура, $N$                & $\pgfmathprintnumber[]{\N}$     \\
		Діаметр контура, $d$, м                  & $\pgfmathprintnumber[]{\D}$     \\
		Струм в котушках Гельмгольца, $I$, А     & $\pgfmathprintnumber[]{\IG}$    \\
		Орієнтація контура, $\alpha$, ${}^\circ$ & $\pgfmathprintnumber[]{\angle}$ \\ \bottomrule
	\end{tabular}
	\label{tab:ExpConst}
\end{table}

Для даного досліду, константу котушок можна обчислити, знаючи коефіцієнт лінійної апроксимації графіка $M = M(I')$:
\begin{equation}\label{eq:TorqueEk}
	M = k I'.
\end{equation}

Виходячи з формули~\eqref{eq:TorqueE} та даних таблиці~\ref{tab:ExpConst} бачимо, що
\begin{equation}\label{eq:Ck}
	C = \frac{4k}{IN\pi d^2 \sin\alpha} = \pgfmathparse{4/(\IG*\N*pi*\D^2*sin(\angle))}\pgfmathprintnumber\pgfmathresult \cdot k.
\end{equation}

В експериментах було виміряно момент сили, що діє на контур зі струмом в магнітному полі. Оскільки динамометр прилада градуйований в мН, то для отримання значення моменту, вказані значення сили необхідно домножити на плече, що дорівнює $12$~см. Результати занесені до таблиці~\ref{tab:expresults}

%========================= Завантаження данних з таблиці ==========================%
%----- Присвоєння данних змінній \TorqueVsCurrentCoilTable
\pgfplotstableread{
	Current CError Force    FError
	0       0        0          0
	0.50    0.012    0.40e-03   0.05e-3
	0.80    0.012    0.60e-03   0.05e-3
	1.00    0.012    0.70e-03   0.05e-3
	1.50    0.012    1.00e-03   0.05e-3
	1.90    0.012    1.30e-03   0.05e-3
	1.70    0.012    1.20e-03   0.05e-3
	1.20    0.012    0.75e-03   0.10e-3
}\TorqueVsCurrentCoilTable
%=============== Додавання нової колонки в таблицю =========================%
%----- Додається нова колонка Torque (обертовий момент), значення в якій є результатом множення колонки Force (Сила) на 0.12
\pgfplotstablecreatecol[
	create col/expr={\thisrow{Force}*0.12},
]{Torque}\TorqueVsCurrentCoilTable

%================== Додавання нової колонки в таблицю=======================%
%----- Додається нова колонка TError (похибка для обертового моменту), значення в якій є результатом множення колонки FError (похибки для сили) на 0.12
\pgfplotstablecreatecol[
	create col/expr={\thisrow{FError}*0.12},
]{TError}\TorqueVsCurrentCoilTable

%============================================================================%

%======================= Верстка таблиці  ==================%
\begin{table}[h!]
	\centering
	\caption{Результати вимірювань}
	\label{tab:expresults}
	\pgfplotstabletypeset[
		col sep=space,
		%1000 sep={,},
		every head row/.style={
				before row={\toprule},
				after row={\midrule}
			},
		every last row/.style={after row=\bottomrule},
		empty cells with={-},
		columns/Current/.style={fixed zerofill, dec sep align, sci precision=2, column name={$I'$, А}, },
		columns={Current,Force,Torque},
		columns/Force/.style={fixed zerofill, sci precision=1, column name={$F$, $\cdot 10^{-3}$ Н}, multiply by=1000},
		columns/Torque/.style={fixed zerofill, sci precision=1, column name={$M$, $\cdot 10^{-4}$ Н$\cdot$м}, multiply by=1e4},
	]\TorqueVsCurrentCoilTable
\end{table}

%================================================================================%

За даними результатами побудовано графік~\ref{plt:expresults}.

\begin{tornpage}
	\begin{center}%[h!]
		%\centering
		\begin{tikzpicture}
			\begin{axis}[%
					LabPlotGrid,
					xlabel={$I'$, А},
					ylabel=\empty,
					every y tick scale label/.style={at={(0.05,1)},anchor=south},
					ytick scale label code/.code={$M$, $\cdot 10^{#1}$ Н$\cdot$м},
					legend pos = north west,
					width=1\linewidth,
					height=0.6\linewidth,
				]
				%---- Побудова графіку за даними ----%
				\addplot[%
					blue,
					only marks,
					error bars/.cd,
					y dir = both,  y explicit,
				]
				table[%
						x=Current,
						y = Torque,
						x error = CError,
						y error = TError,
					]\TorqueVsCurrentCoilTable;
				%---- Додавання легенди до вищепобудованого графіку ----%
				\addlegendentry{Експериментальні дані}
				%---- Побудова лінійної апроксимації до даних файлу ----%	
				\addplot[red] table[x=Current, y={create col/linear regression={y = Torque}}]\TorqueVsCurrentCoilTable;
				%---- Параметри регресії ----%
				\xdef\slope{\pgfplotstableregressiona}
				\xdef\ycepte{\pgfplotstableregressionb}
				%---- Додавання легенди до вищепобудованого графіку ----%
				\addlegendentry{$M = \pgfmathprintnumber{\slope} \cdot I \pgfmathprintnumber[print sign]{\ycepte}$ Апроксимація}
			\end{axis}
		\end{tikzpicture}
		\captionof{figure}{Графік залежності обертального моменту від струму у витку $M(I')$.}
		\label{plt:expresults}
	\end{center}
\end{tornpage}

З лінійної апроксимації результатів (рис.~\ref{plt:expresults}) бачимо, що коефіцієнт нахилу $k = \pgfmathprintnumber{\slope}$~Н/А, а значення константи котушок можна розрахувати з~\eqref{eq:Ck}

%----------------- Математичні обчислення ------------%
\fputrue
\pgfmathsetmacro{\CGEI}{4*\slope/(\IG*\N*\D^2*pi*sin(\angle))}% ----- Обчислення значення
\pgfmathfloatparsenumber{\CGEI}% ---- Розпарсити число на мантісу m і експонетну e у вигляді: m*10^e
\pgfmathfloattomacro{\pgfmathresult}{\FlagCGEI}{\MantissaCGEI}{\ExponentCGEI}% --- Запис мантиси і експоненти у відповідні змінні
\fpufalse
%-----------------------------------------------------%

\begin{equation*}
	C \approx
	(\pgfmathprintnumber[fixed, precision=1]{\MantissaCGEI} \pm 1.2) \cdot 10^{\ExponentCGEI} \, \text{Тл/А}
\end{equation*}


Відносна похибка:%
\fputrue\pgfmathparse{(1.22e-4/\CGEI)*100}\fpufalse
\begin{equation*}
	\epsilon \approx \pgfmathprintnumber[fixed, precision=0]{\pgfmathresult}~\text{\%}.
\end{equation*}

Відмінність експериментального значення від теоретичного становить \fputrue\pgfmathparse{abs(\CGEI/\CG - 1)*100}\fpufalse$\left( \frac{C_\mathrm{exp}} {C_\mathrm{theor}}-  1\right) \cdot 100\, \text{\%} = \pgfmathprintnumber\pgfmathresult$~\%.

\subsection{Дослід 2. Залежність обертального моменту від сили струму в \\котушках Гельмгольца}

\pgfplotstableread{
	Current Force FError
	0.50 0.15e-03 0.05e-3
	0.75 0.30e-03 0.05e-3
	0.90 0.35e-03 0.05e-3
	0.95 0.40e-03 0.05e-3
	1.00 0.50e-03 0.05e-3
	1.10 0.55e-03 0.05e-3
	1.20 0.60e-03 0.05e-3
	1.30 0.65e-03 0.05e-3
	1.40 0.65e-03 0.05e-3
	1.50 0.70e-03 0.05e-3
	1.60 0.75e-03 0.05e-3
	1.70 0.80e-03 0.05e-3
	1.80 0.85e-03 0.05e-3
	1.90 0.90e-03 0.05e-3
	2.00 0.95e-03 0.05e-3
	2.10 1.00e-03 0.05e-3
	2.20 1.05e-03 0.05e-3
	2.30 1.10e-03 0.05e-3
	2.40 1.15e-03 0.05e-3
	2.50 1.20e-03 0.05e-3
	2.75 1.40e-03 0.05e-3
}\TorqueVsCurrentHelmholtzTable

%=============== Додавання нової колонки в таблицю =========================%
%----- Додається нова колонка Torque (обертовий момент), значення в якій є результатом множення колонки Force (Сила) на 0.12
\pgfplotstablecreatecol[
	create col/expr={\thisrow{Force}*0.12},
]{Torque}\TorqueVsCurrentHelmholtzTable

%================== Додавання нової колонки в таблицю=======================%
%----- Додається нова колонка TError (похибка для обертового моменту), значення в якій є результатом множення колонки FError (похибки для сили) на 0.12
\pgfplotstablecreatecol[
	create col/expr={\thisrow{FError}*0.12},
]{TError}\TorqueVsCurrentHelmholtzTable


\subsection{Дослід 3. Залежність обертального моменту від орієнтації контура}

\pgfplotstableread{
	a Force FError
	15 0.10e-3 0.05e-3
	30 0.60e-3 0.05e-3
	45 1.00e-3 0.05e-3
	60 1.25e-3 0.05e-3
	75 1.40e-3 0.05e-3
	90 1.50e-3 0.05e-3
}\TorqueVsSinAlphaTable

\pgfplotstablecreatecol[
	create col/expr={\thisrow{Force}*0.12},
]{Torque}\TorqueVsSinAlphaTable

\pgfplotstablecreatecol[
	create col/expr={sin(\thisrow{a})},
]{sina}\TorqueVsSinAlphaTable

\pgfplotstablecreatecol[
	create col/expr={\thisrow{FError}*0.12},
]{TError}\TorqueVsSinAlphaTable

\begin{tornpage}
	\begin{center}%[h!]
		%\centering
		\begin{tikzpicture}
			\begin{axis}[%
					LabPlotGrid,
					xlabel={$\sin\alpha$},
					ylabel=\empty,
					every y tick scale label/.style={at={(0.05,1)},anchor=south},
					ytick scale label code/.code={$M$, $\cdot 10^{#1}$ Н$\cdot$м},
					legend pos = north west,
					width=1\linewidth,
					height=0.6\linewidth,
				]
				%---- Побудова графіку за даними ----%
				\addplot[%
					blue,
					only marks,
					error bars/.cd,
					y dir = both,  y explicit,
				]
				table[%
						x=sina,
						y = Torque,
						y error = TError,
					]\TorqueVsSinAlphaTable;
				%---- Додавання легенди до вищепобудованого графіку ----%
				\addlegendentry{Експериментальні дані}
				%---- Побудова лінійної апроксимації до даних файлу ----%	
				\addplot[red] table[x=sina, y={create col/linear regression={y = Torque}}]\TorqueVsSinAlphaTable;
				%---- Параметри регресії ----%
				\xdef\slope{\pgfplotstableregressiona}
				\xdef\ycepte{\pgfplotstableregressionb}
				%---- Додавання легенди до вищепобудованого графіку ----%
				\addlegendentry{$M = \pgfmathprintnumber{\slope} \cdot \sin\alpha \pgfmathprintnumber[print sign]{\ycepte}$ Апроксимація}
			\end{axis}
		\end{tikzpicture}
		\captionof{figure}{Графік залежності обертального моменту від $\sin\alpha$ у витку $M(\sin\alpha)$.}
		\label{plt:expresults}
	\end{center}
\end{tornpage}

З лінійної апроксимації результатів (рис.~\ref{plt:expresults}) бачимо, що коефіцієнт нахилу $k = \pgfmathprintnumber{\slope}$~Н/А, а значення константи котушок можна розрахувати з~\eqref{eq:Ck}


\pgfmathsetmacro{\N}{3}
\pgfmathsetmacro{\D}{0.12}
\pgfmathsetmacro{\IG}{2.85}
\pgfmathsetmacro{\I}{2}

%----------------- Математичні обчислення ------------%
\fputrue
\pgfmathsetmacro{\CGEsin}{4*\slope/(\IG*\I*\N*\D^2*pi)}% ----- Обчислення значення
\pgfmathfloatparsenumber{\CGEsin}% ---- Розпарсити число на мантісу m і експонетну e у вигляді: m*10^e
\pgfmathfloattomacro{\pgfmathresult}{\FlagCGEsin}{\MantissaCGEsin}{\ExponentCGEsin}% --- Запис мантиси і експоненти у відповідні змінні
\fpufalse
%-----------------------------------------------------%

\begin{equation*}
	C \approx
	(\pgfmathprintnumber[fixed, precision=1]{\MantissaCGEsin} \pm 0.6) \cdot 10^{\ExponentCGEI} \, \text{Тл/А}
\end{equation*}

\subsection{Дослід 4. Залежність обертального моменту від площі контура}

\pgfplotstableread{
	d Force FError
	0.06 0.10e-3 0.05e-3
	0.09 0.15e-3 0.05e-3
	0.12 0.25e-3 0.05e-3
}\TorqueVsDiameterTable

\pgfplotstablecreatecol[
	create col/expr={\thisrow{Force}*0.12},
]{Torque}\TorqueVsDiameterTable

\pgfplotstablecreatecol[
	create col/expr={\thisrow{FError}*0.12},
]{TError}\TorqueVsDiameterTable

\pgfplotstablecreatecol[
	create col/expr={\thisrow{d}^2},
]{ds}\TorqueVsDiameterTable

\subsection{Дослід 5. Залежність обертального моменту від кількості витків \\контура}

\pgfplotstableread{
	N Force FError
	1 0.25e-3 0.05e-3
	2 0.80e-3 0.05e-3
	3 1.50e-3 0.05e-3
}\TorqueVsNumderTable

\pgfplotstablecreatecol[
	create col/expr={\thisrow{Force}*0.12},
]{Torque}\TorqueVsNumderTable

\pgfplotstablecreatecol[
	create col/expr={\thisrow{FError}*0.12},
]{TError}\TorqueVsNumderTable


\section{Обговорення результатів}


\section*{Висновки}

\end{document}
