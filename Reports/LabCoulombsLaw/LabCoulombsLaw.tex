% !TeX program = lualatex
% !TeX encoding = utf8
% !TeX spellcheck = uk_UA
% !BIB program = biber

\documentclass{LabWork}
\usepackage{fixfoot}
%\DeclareFixedFootnote{\repnote}{\fullcite{Spavieri2004}}
%\addbibresource{LabWork1.bib}
\graphicspath{{LabWork1pic/}}
\usetikzlibrary{arrows.meta}
\tikzset{
every info/.style={font=\small},
}

%============================================= Заголовок документу ====================================================%
\work{1}
\title{Перевірка закону Кулона}

\author{Тор А.~В.}{}
\author{Другий А.~В.}{}

\group{ФФ-93}

\abstract{%

Перевірити закон Кулона:
\begin{enumerate}
\item Визначити залежність сили від заряду;
\item визначити залежність сили від відстані;
\item визначити електричну сталу в системі SI.
\end{enumerate}
}
%======================================================================================================================%

\begin{document}
\writedatatofile{\jobname}
\maketitle

\section{Експериментальні результати та їх обробка}

Результати досліджень занесемо до табл.~\ref{tab:tableQF}.

\pgfplotstableread{QF.dat}\QFTable
\pgfplotstablecreatecol[
	create col/expr={\thisrow{Q1}^2},
]{Qsquare1}\QFTable
\pgfplotstablecreatecol[
	create col/expr={\thisrow{Q2}^2},
]{Qsquare2}\QFTable
\pgfplotstablecreatecol[
	create col/expr={\thisrow{Q3}^2},
]{Qsquare3}\QFTable
\pgfplotstablecreatecol[
	create col/expr={\thisrow{Q4}^2},
]{Qsquare4}\QFTable
\pgfplotstablecreatecol[
	create col/expr={\thisrow{Q5}^2},
]{Qsquare5}\QFTable

\begin{table}[hb!]\centering\small
\caption{Результати дослідження}
\label{tab:tableQF}
\pgfplotstabletypeset[
    clear infinite,
    every head row/.style={
        before row={
        \toprule
        \multicolumn{2}{c}{$a = 4$~см} & \multicolumn{2}{c}{$a = 5$~см} & \multicolumn{2}{c}{$a = 6$~см} & \multicolumn{2}{c}{$a = 7$~см} & \multicolumn{2}{c}{$a = 8$~см}\\
        },
    after row=\midrule,
    },
    every last row/.style={
    after row=\bottomrule},
    columns/Q1/.style={
        column name={$q$, нКл},
        fixed,fixed zerofill,
        multiply by=1e9,
        precision=1},
    columns={Q1,F1,Q2,F2,Q3,F3,Q4,F4,Q5,F5},
    columns/F1/.style={
        column name={$F$, мН},
        fixed,fixed zerofill,
        multiply by=1e3,
        precision=2},
    columns/Q2/.style={
        column name={$q$, нКл},
        fixed,fixed zerofill,
        multiply by=1e9,
        precision=1},
    columns/F2/.style={
        column name={$F$, мН},
        fixed,fixed zerofill,
        multiply by=1e3,
        precision=2},
    columns/Q3/.style={
        column name={$q$, нКл},
        fixed,fixed zerofill,
        multiply by=1e9,
        precision=1},
    columns/F3/.style={
        column name={$F$, мН},
        fixed,fixed zerofill,
        multiply by=1e3,
        precision=2},
    columns/Q4/.style={
        column name={$q$, нКл},
        fixed,fixed zerofill,
        multiply by=1e9,
        precision=1},
    columns/F4/.style={
        column name={$F$, мН},
        fixed,fixed zerofill,
        multiply by=1e3,
        precision=2},
    columns/Q5/.style={
        column name={$q$, нКл},
        fixed,fixed zerofill,
        multiply by=1e9,
        precision=1},
    columns/F5/.style={
        column name={$F$, мН},
        fixed,fixed zerofill,
        multiply by=1e3,
        precision=2},
]\QFTable
\end{table}

На основі результатів дослідження (табл.~\ref{tab:tableQF}), побудуємо графіки (рис.~\ref{plt:expresults}).
\begin{tornpage}
	\begin{center}%[h!]
		%\centering
		\begin{tikzpicture}[%
            declare function={ F(\A,\s,\B) = \A*x^(1 + \s/2) + \B;  },
            ]
			\begin{axis}[%
					LabPlotGrid,
					xlabel={$q^2$, Кл$^2$},
					ylabel=\empty,
					every y tick scale label/.style={at={(0.05,1)},anchor=south},
					ytick scale label code/.code={$F$, $\cdot 10^{#1}$ Н},
					legend pos = north west,
					width=1\linewidth,
					height=0.6\linewidth,
                    xmin=0,
				]
				%---- Побудова графіку за даними ----%
				\addplot[%
					blue,
					only marks,
                    mark=*,
					error bars/.cd,
					y dir = both,  y explicit,
				]
				table[%
						x=Qsquare1,
						y = F1,
%						x error = CError,
%						y error = TError,
					]\QFTable;
   				\addlegendentry{$a = 4$~см}
				\addplot[%
					blue,
					only marks,
                    mark=triangle*,
					error bars/.cd,
					y dir = both,  y explicit,
				]
				table[%
						x=Qsquare2,
						y = F2,
%						x error = CError,
%						y error = TError,
					]\QFTable;
   				\addlegendentry{$a = 5$~см}
				\addplot[%
					blue,
					only marks,
                    mark=diamond*,
					error bars/.cd,
					y dir = both,  y explicit,
				]
				table[%
						x=Qsquare3,
						y = F3,
%						x error = CError,
%						y error = TError,
					]\QFTable;
   				\addlegendentry{$a = 6$~см}
				\addplot[%
					blue,
					only marks,
                    mark=square*,
					error bars/.cd,
					y dir = both,  y explicit,
				]
				table[%
						x=Qsquare4,
						y = F4,
%						x error = CError,
%						y error = TError,
					]\QFTable;
   				\addlegendentry{$a = 7$~см}
				\addplot[%
					blue,
					only marks,
                    mark=o,
					error bars/.cd,
					y dir = both,  y explicit,
				]
				table[%
						x=Qsquare5,
						y = F5,
%						x error = CError,
%						y error = TError,
					]\QFTable;
   				\addlegendentry{$a = 8$~см}

    \addplot [no markers, red] gnuplot [raw gnuplot] {
            f(x) = A*x**(1 + s/2) + B;  
            A=1e+12; s=1e-4; B = 1e-6 ; 
            set print "A.dat"; 
            print "a A dA s"; 
            fit f(x) "QF.dat" u ($1**2):2 via A,B; plot  [x=0:6.5e-16] f(x); print 4e-2,A,A_err,s;
            fit f(x) "QF.dat" u ($3**2):4 via A,B; plot  [x=0:6.5e-16] f(x); print 5e-2,A,A_err,s;
            fit f(x) "QF.dat" u ($5**2):6 via A,B; plot  [x=0:6.5e-16] f(x); print 6e-2,A,A_err,s;
            fit f(x) "QF.dat" u ($7**2):8 via A,B; plot  [x=0:6.5e-16] f(x); print 7e-2,A,A_err,s;
            fit f(x) "QF.dat" u ($9**2):10 via A,B; plot  [x=0:6.5e-16] f(x); print 8e-2,A,A_err,s;
            set print;
    };



			\end{axis}
		\end{tikzpicture}
		\captionof{figure}{Графік залежності $F$ від  $q^2$.}
		\label{plt:expresults}
	\end{center}
\end{tornpage}

\pgfplotstableread{A.dat}\ATable
\pgfplotstablecreatecol[
	create col/expr={1/\thisrow{a}^2},
]{inva2}\ATable

Для більш точної нелінійної апроксимації за формулою $F = A \cdot q^2 \cdot q^{\sigma/2} + B$, будемо підбирати показник степеря $\sigma$ самостійно, домагаючись найменшої похибки в значенні параметра $A$. В теорії, параметр $B = 0$, однак в нашому випадку він може містити інформацію про додаткові сили пружності з боку дрота, яким наша кулька під'єднана до джерела високої напруги та інші невраховані сили, які дають внесок в систематичну похибку.

З результатів апроксимації за формулою отримуємо наступні дані включені до табл.~\ref{tab:results}. Показник степеня $\sigma = 1 \cdot 10^{-4}$.

\begin{table}[ht!]\centering
\caption{Результати апроксимації графіків~\ref{plt:expresults}}
\label{tab:results}
\pgfplotstabletypeset[
    every head row/.style={
        before row={
        \toprule
        },
    after row=\midrule,
    },
    every last row/.style={
    after row=\bottomrule},
    columns={inva2, A, dA, s},
%    columns/a/.style={
%        column name={$a$, м},
%        sci,sci zerofill,
%        precision=1},
    columns/inva2/.style={
        column name={$1/a^2$, м$^{-2}$},
        sci zerofill,
        precision=0},
    columns/A/.style={
        fixed,fixed zerofill,
        multiply by=1e-12,
        column name={$A = F/q^2$, $10^{12}~$Н/Кл$^{2}$},
        precision=2},
    columns/dA/.style={
        column name={$\Delta(F/q^2)$, $10^{12}~$Н/Кл$^{2}$},
        fixed,fixed zerofill,
        multiply by=1e-12,
        precision=2},
    columns/s/.style={
        column name={$\sigma$},
        sci, sci zerofill,
        precision=0},
]\ATable
\end{table}

На основі результатів апроксимації (табл.~\ref{tab:results}), побудуємо графіки (рис.~\ref{plt:expresults2}).

\begin{tornpage}
	\begin{center}%[h!]
		%\centering
		\begin{tikzpicture}
			\begin{axis}[%
					LabPlotGrid,
					xlabel={$1/{a^2}$, м$^{-2}$},
					ylabel=\empty,
					every y tick scale label/.style={at={(0.05,1)},anchor=south},
					ytick scale label code/.code={$F/q^2$, $\cdot 10^{#1}$ Н/Кл$^{2}$},
					legend pos = north west,
					width=1\linewidth,
					height=0.6\linewidth,
                    xmin=0, ymin=0,
				]
				%---- Побудова графіку за даними ----%
				\addplot[%
					blue,
					only marks,
					error bars/.cd,
					y dir = both,  y explicit,
				]
				table[%
						x=inva2,
						y = A,
%						x error = CError,
						y error = dA,
				]\ATable;
				%---- Додавання легенди до вищепобудованого графіку ----%
				\addlegendentry{Експериментальні дані}
				%---- Побудова лінійної апроксимації до даних файлу ----%	
%				\addplot[red] table[x=Current, y={create col/linear regression={y = Torque}}]\QFTable;
    \addplot [no markers, red] gnuplot [raw gnuplot] { % "raw gnuplot" allows us to use arbitrary gnuplot commands
            f(x) = k/4*x**(1+e/2) + C;  % Define the function to fit
            k=2.8e+9; e=2e-2; C = -0.1e12 ;     % Set reasonable starting values here
            fit f(x) "A.dat" u (1/$1**2):2 via k,C; 
            plot  [x=150:650] f(x); % Specify the range to plot
%            plot  "A.dat" u (1/$1**2):2; % Specify the range to plot
    };
				%---- Параметри регресії ----%
%				\xdef\slope{\pgfplotstableregressiona}
%				\xdef\ycepte{\pgfplotstableregressionb}
%				%---- Додавання легенди до вищепобудованого графіку ----%
			\end{axis}
		\end{tikzpicture}
		\captionof{figure}{Графік залежності ${F}/{q^2}$ від $1/{a^2}$.}
		\label{plt:expresults2}
	\end{center}
\end{tornpage}

Результати апроксимації графіка \ref{plt:expresults2} за формулою $A = \frac{k}{4} \cdot (1/a^2) \cdot (1/a^2)^{\varepsilon/2}$ дають отримане значення $\varepsilon = 2\cdot 10^{-2}$. Аналогічно, ми підбирали показник степеря $\varepsilon$ самостійно, домагаючись найменшої похибки для $k$.

Для значення константи $k$ результати апроксимації дають:
\begin{equation}\label{}
    k = (10.5 \pm 0.5)\cdot 10^{9}\ \frac{\text{Н}\cdot\text{м}^2}{\text{Кл}^2}
\end{equation}

Табличне прецизійне значення
\begin{equation}\label{}
    k = 8.9875517923(14) \cdot 10^{9}\ \frac{\text{Н}\cdot\text{м}^2}{\text{Кл}^2}
\end{equation}

Таким чином, відмінність нашого значення від табличного становить $\approx 17\%$.

\section*{Висновки}

В результаті дослідів було перевірено закон Кулона. Встановлено, що сила взаємодії однойменних прямо пропорційна квадрату $F \propto q^{2 + \sigma}$. Оцінка $\sigma $ дає величину: 
\[
    \sigma \le 1\cdot10^{-4}.
\]

Також було перевірено <<закон обернених квадратів>> $F \propto \frac{1}{r^{2 + \varepsilon}}$. Оцінка $ \varepsilon $ дає величину: 
\[
    \varepsilon \le 2\cdot10^{-2}.
\]

Для значення константи $k$ результати апроксимації дають:
\begin{equation*}\label{}
    k = (10.5 \pm 0.5)\cdot 10^{9}\ \frac{\text{Н}\cdot\text{м}^2}{\text{Кл}^2}
\end{equation*}

Табличне прецизійне значення
\begin{equation*}\label{}
    k = 8.9875517923(14) \cdot 10^{9}\ \frac{\text{Н}\cdot\text{м}^2}{\text{Кл}^2}
\end{equation*}

Відмінність нашого значення від табличного становить $\approx 17\%$.

Для перевірки закону Кулона в нашій роботі ми користувались прямим методом, який має низьку точність, оскільки ефекти електростатичної індукції призводять до того, що заряди наводяться практично на всіх тілах, що
оточуючих використовуваний прилад, а також, ці ефекти призводять до того, що заряд на кулькі розподілений нерівномірно. На точність також впливає і явище стікання електричного заряду з кульки. Однак, основна похибка обумовлена тим, що коромисло не вдається встановити в початкове положення, тобто визначити, коли нитка не закручена, бо цьому перешкоджає рух повітря навколо кульки.

Для більш точної перевірки закону <<обернених квадратів>> сьогодні користуються непрямими методами, які полягають у вимірювання електричних потенціалів на концентричних сферах.
\end{document}
