% !TeX program = lualatex
% !TeX encoding = utf8
% !TeX spellcheck = uk_UA
% !BIB program = biber



\documentclass{LabWork}
\usepackage{lipsum}


\usepackage{minted}
%\DeclareFixedFootnote{\repnote}{\fullcite{Spavieri2004}}
%\addbibresource{LabWork1.bib}
\graphicspath{{LabWork1pic/}}
\usetikzlibrary{arrows.meta}
\tikzset{
every info/.style={font=\small},
}

%====================== Заголовок документу ==========================%
\work{2}
\title{Робота в системі \LaTeX}

\author{Красильнікова А.}{} % в класі LabWork.cls визначена команда \author{#1}{#2} з двома параметрами. Параметр #2 не задіяний, тому пустий. Але команда \author чекає, що для нього будуть дужки

\group{ФФ-11}

\abstract{%
Навчитись користуватись \LaTeX
}
%=====================================================================%

\begin{document}
\maketitle

\section{Експериментальні результати}



\section{Обробка результатів}


\begin{figure}[h!]\centering
    \begin{tikzpicture}

        \begin{axis}[width = 15cm,
        legend style = {font = \small,
            fill opacity = 0.5,
            text opacity = 1,
            draw = none,
            cells = {align = left, anchor = west},
            pos = south east},
        grid = both,
        grid style = {line width=.1pt},
        major grid style = {line width=.2pt},
        minor grid style = {line width=.1pt},
        tick label style = {font=\sffamily},
        tick align = inside,
        legend style = {font = \sffamily},
        xlabel = {$t$, с},
        ylabel = {$y$, м}]

            \addplot [only marks, mark size = 3] plot [error bars/.cd, y dir=both, y explicit] table [y error = spos] {lab.txt};

           \addlegendentry{експеримент}

            \addplot[domain=0:2.2, red]   {-4.9*x^2+12.5*x+0.16}; % 4,9 -> 4.9

            \addlegendentry{апроксимація $y(t) = -4.9t^2 +12.5t + 0.16$ }

        \end{axis}
    \end{tikzpicture}
    \caption{y(t)}
    \label{pic: y(t)}
\end{figure}

\newpage



\section*{Висновки}


\end{document}
