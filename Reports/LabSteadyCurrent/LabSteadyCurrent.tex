% !TeX program = lualatex
% !TeX encoding = utf8
% !TeX spellcheck = uk_UA
% !BIB program = bibtex8

\documentclass{LabWork}
\usetikzlibrary{arrows.meta}
\tikzset{
every info/.style={font=\small},
}

%============================================= Заголовок документу ====================================================%
\work{4}
\title{Джерела електричної енергії та їх характеристики}

\author{Тор А.~В.}{}
\author{Другий А.~В.}{}

\group{ФФ-93}

\abstract{%

Виміряти напругу $U$, на клемах декількох джерел електроенергії як функцію сили струму, змінюючи зовнішній опір $R_e$, і розрахувати напругу без навантаження $\mathcal{E}$ і внутрішній опір $R_i$ джерела.

Виміряти безпосередньо напругу без навантаження від джерела електроенергії (без зовнішнього опору) та його внутрішній опір (підбираючи умови узгодження $R_i  =  R_e$). Визначити силову діаграму залежності між напругою на клемах та силою струму.
}
%======================================================================================================================%

\begin{document}
\writedatatofile{\jobname}
\maketitle

\section{Експериментальні дані}

\pgfplotstableread{
	I	U
	0.4	8.6
	0.5	8.3
	0.6	7.9
	0.7	7.3
	0.8	6.6
	0.9	6.3
	1.0	5.8
	1.1	5.2
	1.2	4.5
	1.3	4.1
	1.4	3.6
	1.5	3.1
}\UvsI

\pgfplotstablecreatecol[
	create col/expr={0.0125/\thisrow{I}},
]{Ierror}\UvsI

\pgfplotstablecreatecol[
	create col/expr={0.5/\thisrow{U}},
]{Uerror}\UvsI

\begin{table}[!htbp]\centering
	\caption{Таблиця з експериментальними даними кислотно-свинцевого акумулятора}
	\label{tab1}
	\pgfplotstabletypeset[
		multicolumn names,
		columns/I/.style={
				column name={$I$, А},
				column type={
						S[
								round-mode=places,
								round-precision=2,
								%                scientific-notation=engineering,
								%                table-format=2.1,
								exponent-product = \cdot
							]
					},
				string type
			},
		columns/U/.style={
				column name={$U$, В},
				column type={
						S[
								round-mode=places,
								round-precision=2,
								%                scientific-notation=engineering,
								%                table-format=2.1,
								exponent-product = \cdot
							]
					},
				string type
			},
		columns/Ierror/.style={
				column name={$\Delta I$, А},
				fixed zerofill, fixed,
			},
		columns/Uerror/.style={
				column name={$\Delta U$, В},
				fixed zerofill, fixed,
			},
		every head row/.style={
				before row={\toprule},
				after row={\midrule},
			},
		every last row/.style={after row=\bottomrule}
	]\UvsI
\end{table}

\begin{tornpage}
	\begin{center}%[h!]
		%\centering
		\begin{tikzpicture}
			\begin{axis}[legend style={font=\scriptsize},
					legend pos = north east,
					xlabel={$I$, А},
					ylabel={$U$, В},
					%				every axis y label/.style={
					%				    at={(yticklabel cs:1)},
					%				    anchor=south,
					%				},
					%				every y tick scale label/.style={at={(0.05,1)},anchor=south},
					ytick scale label code/.code={$U$, $\cdot 10^{#1}$ Н$\cdot$м},
					%	xmin = 3,
					%	xmax = 8,
					minor tick num = 1,
					width=1\linewidth,
					height=0.6\linewidth,
					% === Налаштування сітки ===
					grid = both,
					major grid style={line width=.6pt,draw=brown!60},
					minor tick num = 9,
					minor grid style = {line width=.1pt,draw=brown!20},
				]

				%-------------------------------------------- Побудова графіку за даними з файлу -----------------------------------------%
				\addplot[
					blue,
					only marks,
					error bars/.cd,
					x dir=both, x explicit,
					y dir=both, y explicit,
				]
				table[
						x= I,
						x error = Ierror,
						y = U,
						y error = Uerror,
					]\UvsI;

				%------------------------------------ Додавання легенди до вищепобудованого графіку --------------------------------------%
				\addlegendentry{Експериментальні дані}

				%---------------------------------- Побудова лінійної апроксимації до даних файлу ----------------------------------------%	
				\addplot[red,
				]
				table[x=I, y={create col/linear regression={y = U}}]\UvsI;
				\xdef\slope{\pgfplotstableregressiona}
				\xdef\ycepte{\pgfplotstableregressionb}

				%------------------------------------ Додавання легенди до вищепобудованого графіку --------------------------------------%
				\addlegendentry{
					$U = \pgfmathprintnumber{\slope} \cdot I \pgfmathprintnumber[print sign]{\ycepte}$
					Апроксимація
				}
			\end{axis}
		\end{tikzpicture}
		\captionof{figure}{Графік залежності $U(I)$ за даними табл.~\ref{tab1}}
		\label{plt:expresults}
	\end{center}
\end{tornpage}

\pgfmathsetmacro{\Uz}{10.86}
\pgfmathsetmacro{\Ri}{5.18}
\pgfmathsetmacro{\Iz}{\Uz/\Ri}

\pgfplotstablecreatecol[
	create col/expr={\thisrow{U}/\thisrow{I}/\Ri},
]{ReRi}\UvsI

\pgfplotstablecreatecol[
	create col/expr={\thisrow{U}/\Uz},
]{UU0}\UvsI

\pgfplotstablecreatecol[
	create col/expr={\thisrow{I}/\Iz},
]{II0}\UvsI

\pgfplotstablecreatecol[
	create col/expr={\thisrow{U}/\Uz*\thisrow{I}/\Iz},
]{PP0}\UvsI

\begin{tornpage}
	\begin{center}%[ht!]\centering
		\begin{tikzpicture}
			\begin{semilogxaxis}[
					x label style={at={(axis description cs:0.5,-0.1)},anchor=north},
					legend style={row sep = 5pt},
					xlabel={$\frac{R_e}{R_i}$},
					width = \linewidth,
					height = 0.8\linewidth,
					grid = both,
					major grid style={line width=.6pt,draw=brown!60},
					minor tick num = 9,
					minor grid style = {line width=.1pt,draw=brown!20},
					axis lines=center,
					%        axis y line shift=1,
				]

				\addplot[domain=0.01:100, samples=100, blue] {1 - 1/(1 + x/1)};
				\addplot[domain=0.01:100, samples=100, green!50! black] {1/(1 + x/1)};
				\addplot[domain=0.01:100, samples=100, red] {x/1/(1+x/1)^2};

				\addplot[
					blue,
					only marks,
					error bars/.cd,
					y dir = both,  y explicit,
				]
				table[
						x= ReRi,
						y = UU0,
					]\UvsI;

				\addplot[
					green!50! black,
					only marks,
					error bars/.cd,
					y dir = both,  y explicit,
				]
				table[
						x= ReRi,
						y = II0,
					]\UvsI;

				\addplot[
					red,
					only marks,
					error bars/.cd,
					y dir = both,  y explicit,
				]
				table[
						x= ReRi,
						y = PP0,
					]\UvsI;


				\legend{${U}/{\mathcal{E}}$, ${I}/{I_0}$, ${P}/{P_0}$}
			\end{semilogxaxis}
		\end{tikzpicture}
		\captionof{figure}{Діаграма режимів для акумулятора}
		\label{diag}
	\end{center}
\end{tornpage}


\end{document}
