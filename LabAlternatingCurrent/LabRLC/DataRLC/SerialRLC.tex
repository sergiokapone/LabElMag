 % !TeX program = lualatex
% !TeX encoding = utf8
% !TeX spellcheck = uk_UA
% !BIB program = bibtex8

\documentclass[]{article}

\usepackage{fontspec}
\setsansfont{CMU Sans Serif}%{Arial}
\setmainfont{CMU Serif}%{Times New Roman}
\setmonofont{CMU Typewriter Text}%{Consolas}
\defaultfontfeatures{Ligatures={TeX}}
\usepackage[math-style=TeX]{unicode-math}
\usepackage[english, russian, ukrainian]{babel}
\usepackage[%
	a4paper,%
	footskip=1cm,%
	headsep=0.3cm,% 
	top=1cm, %поле сверху
	bottom=1cm, %поле снизу
	left=1cm, %поле ліворуч
	right=1cm, %поле праворуч
    ]{geometry}

\usepackage{tikz}
\usetikzlibrary{shadows}
\usepackage[many, most]{tcolorbox}
\newtcolorbox{tornpage}{%
    enhanced jigsaw, % allow page breaks
    frame hidden, % hide the default frame
    overlay={%
        \draw [
            fill=white, % fill paper
            draw=white!50!black, % boundary colour
            decorate, % decoration
            decoration={random steps,segment length=2pt,amplitude=1pt},
            drop shadow, % shadow
        ]
        % top line
        (frame.north west)--(frame.north east)--
        % right line
        (frame.north east)--(frame.south east)--
        % bottom line
        (frame.south east)--(frame.south west)--
        % left line
        (frame.south west)--(frame.north west);
    },
    % paragraph skips obeyed within tcolorbox
    parbox=false,
}
\usepackage{caption}
\usepackage{pgfplots}
\usetikzlibrary{fpu}
\usetikzlibrary{intersections}
\usetikzlibrary{backgrounds}
\usetikzlibrary{arrows.meta}  
\usetikzlibrary{circuits.ee.IEC}
\usetikzlibrary{intersections}
\tikzset{circuit ee IEC,
every info/.style=red,
semithick,
every info/.style={font=\footnotesize},
small circuit symbols,
circuit declare symbol = ocontact,
circuit declare symbol = ampermeter,
circuit declare symbol = voltmeter,
circuit declare symbol = galvanometer,
set ampermeter graphic = {draw,generic circle IEC, minimum size=5mm,info=center:A},%
set voltmeter graphic = {draw,generic circle IEC, minimum size=5mm,info=center:V},
set galvanometer graphic = {draw,generic circle IEC, minimum size=10mm,info=center:G},
set ocontact graphic = {draw, fill = white, generic circle IEC, shape=circle, minimum size=1mm},
}


\pgfkeys{/pgf/number format/.cd,custom exponent/.initial=2}%
\newcommand{\pgfmathprintnumberFE}[2][]{%
\begingroup
\pgfkeys{/pgf/number format/.cd,fixed,#1}%
\pgfset{fpu=true}%
\pgfmathparse{#2}%
\pgfmathfloattomacro{\pgfmathresult}{\F}{\M}{\E}%
\pgfset{fpu=false}%
\pgfmathtruncatemacro{\redexp}{\E-\pgfkeysvalueof{/pgf/number format/custom exponent}}%
\ifnum\pgfkeysvalueof{/pgf/number format/custom exponent}=0
\ensuremath{\pgfmathprintnumber[#1]{\pgfmathresult}}%
\else
\pgfmathsetmacro{\newnum}{\M*pow(10,\redexp)}%
\ensuremath{\pgfmathprintnumber{\newnum}\cdot10^{\pgfkeysvalueof{/pgf/number format/custom exponent}}}%
\fi
\endgroup}

\begin{document}
\pagestyle{empty}

\newsavebox{\circuit}
\sbox{\circuit}{%
	\begin{tikzpicture}
		\draw (0,-2) coordinate (START) to [resistor={info'={$R_i$}, color=red}] ++(0,2) to [ac source={rotate=-90,info'={right:$\mathcal{E}$}}] (0,2) to [resistor={info'={$R_I$}}] ++(4,0) to [inductor={info'={$L$}}] ++(0,-4/3) to [resistor={info'={$R_{L}$}, color=red}] ++(0,-4/3) to [capacitor={info'={$C$}}] ++(0,-4/3)  -- (START);
		\node [text width=2cm] at (2,0) {\small Послідовний\\ $RLC$-контур};
	\end{tikzpicture}
}

%---------------------------------------------------------
\begin{center}
	\begin{tikzpicture}[
			declare function={
					L  = 2e-3;
					C  = 1.2e-6;
					R  = 10;
					RL  = 0.8;
					Ri  = 5;
					Rtot = R + RL + Ri;
					U  = 10;
					omegares  = 1/sqrt(L*C);
					nures  = omegares/(2*pi);
					Q  = sqrt(L/C)/Rtot;
					Imax  = U/Rtot;
					D1 =  (1/(4*pi*L))*(sqrt( (Rtot^2+4*L/C)) - Rtot);
					D2 = (1/(4*pi*L))*(sqrt( (Rtot^2*C+4*L)/C) + Rtot);
					I(\x) = Rtot/sqrt(Rtot^2 + ((2*pi*\x)*L-1/(C*(2*pi*\x)))^2);
					F2(\x) = sqrt(Rtot^2/(pi^2*\x^2*L^2) + sqrt(((\x^2 - 1)*Rtot^2*(\x^2*C*Rtot^2 - 4*\x^2*L - C*Rtot^2))/(\x^4*C))/(pi^2*L^2) + 2/(pi^2*C*L) - Rtot^2/(pi^2*L^2))/(2*sqrt(2));
					F1(\x) = sqrt(Rtot^2/(pi^2*\x^2*L^2) - sqrt(((\x^2 - 1)*Rtot^2*(\x^2*C*Rtot^2 - 4*\x^2*L - C*Rtot^2))/(\x^4*C))/(pi^2*L^2) + 2/(pi^2*C*L) - Rtot^2/(pi^2*L^2))/(2*sqrt(2));
				},
		]


		\begin{axis}[
				% === Налаштування сітки ===
				grid = both,
				major grid style={line width=.6pt,draw=brown!60},
				minor tick num = 9,
				minor grid style = {line width=.1pt,draw=brown!20},
				% === Налаштування положення координатних осей ===
				axis lines = middle,
				axis line style={-stealth},
				% === Підпис координатних осей ===
				%				xticklabels={},
				xlabel={$\nu$, Hz},
				ylabel={$\frac{I_{0}}{I_{\max}}$},
				% === Положення підпису координатних осей ===
				ylabel style={above right},
				every tick/.style={black},
				extra tick style={% changes for all extra ticks
						tick align=outside,
						major grid style={dashed,draw=black}
					},
			    extra x tick style={
			      red,
			      major tick style={
			        gray,
			      },
			      tick label style={
					yshift=6mm,
			        /pgf/number format/.cd, fixed, fixed zerofill,
			        precision=2,
			      }
			    },
				extra y tick style={% changes for extra y ticks
						tick label style={yshift=2mm},
					},
				extra x ticks={F1(1/sqrt(2)), nures, F2(1/sqrt(2)},
				extra x tick labels={
					\pgfmathprintnumberFE[custom exponent=0]{F1(1/sqrt(2))/100},
					\pgfmathprintnumberFE[custom exponent=0]{nures/100},
					\pgfmathprintnumberFE[custom exponent=0]{F2(1/sqrt(2))/100},
					},
				extra y ticks={1/sqrt(2)},
				extra y tick labels={$1/\sqrt{2}$},
				scaled x ticks=base 10:-2,
				% === Вибір підписів шкали для відображення ===
				xtick = {},
				ytick = {},
				% === Налаштування мінімальних та максимальних значень координат ===
				xmin = 15e2,
				xmax =  65e2,
				ymin = 0.25,
				ymax =  1,
				% === Налаштування розміру графіка ===
				width=1\linewidth,
				height=0.95\textheight,
			]
			\addplot [ultra thick,samples = 500, green!50!black, thick, domain=0.51*nures:1.85*nures, name path global=ResCurve]  {I(x)};


			\pgfplotsinvokeforeach{0.3,0.4,...,0.9}
			{
				\addplot[color=blue,mark=*, only marks, point meta=x, nodes near coords={\pgfmathprintnumberFE[custom exponent=0, precision=1]{\pgfplotspointmeta/100}}, every node near coord/.append style={anchor=west}] coordinates { (F1(#1),#1) (F2(#1),#1) };
			}
			\addplot[color=red,mark=*, only marks, point meta=x, nodes near coords={\pgfmathprintnumberFE[custom exponent=0, precision=1]{\pgfplotspointmeta/100}}, every node near coord/.append style={anchor=south}] coordinates {(F2(1),1)};


			%			\fill [white, draw=green!50!black, name intersections={of=ResCurve and line}] (intersection-1) circle (0.05cm) (intersection-2) circle (0.05cm);
			\node[anchor = north east, text width=5cm] at (current axis.north east) {
				\begin{tornpage}
					$C = \pgfmathprintnumberFE[custom exponent=-6,precision=1]{C}$~Ф,\\
					$L = \pgfmathprintnumberFE[custom exponent=-3,precision=0]{L}$~Гн,\\
					$R_L = \pgfmathprintnumberFE[custom exponent=0,precision=2]{RL}$~Ом,\\
					$R_i = \pgfmathprintnumberFE[custom exponent=0,precision=2]{Ri}$~Ом,\\
					$R_I = \pgfmathprintnumberFE[custom exponent=0,precision=2]{R}$~Ом,\\
					\hrulefill\\
					\medskip%
					$Q^{\mathrm{theor}} = \dfrac{1}{R}\sqrt{\dfrac{L}{C}} = \pgfmathprintnumberFE[custom exponent=0,precision=2]{Q}$\\
					\hrulefill\\
					\smallskip%
					$\nu_0 = \pgfmathprintnumberFE[custom exponent=2,precision=1]{nures}$~Hz \\
					$2\Delta\nu = \pgfmathprintnumberFE[custom exponent=2,precision=1]{D2-D1}$~Hz\\
					\hrulefill\\
					\smallskip%
					$Q^{\mathrm{exp}} = \dfrac{\nu_0}{2\Delta\nu} = \pgfmathprintnumberFE[custom exponent=0,precision=2]{nures/(D2-D1)}$
				\end{tornpage}
			};
		\end{axis}
		\node[fill=white, anchor=east, text width=\wd\circuit + 1cm] at (current axis.east) {
			\begin{tornpage}
				\usebox{\circuit}
			\end{tornpage}
		};
	\end{tikzpicture}
	АЧХ послідовного контуру
\end{center}
%---------------------------------------------------------
\end{document}