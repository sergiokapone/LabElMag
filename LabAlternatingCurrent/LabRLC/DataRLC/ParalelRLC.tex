% !TeX program = lualatex
% !TeX encoding = utf8
% !TeX spellcheck = uk_UA
% !BIB program = bibtex8

\documentclass[tikz, border=1cm]{standalone}
\usepackage{ pgfplots, tikz-3dplot}
\usetikzlibrary{fpu}
\usetikzlibrary{decorations.text, decorations.markings}
\usetikzlibrary{intersections}
\usetikzlibrary{arrows.meta}  
\usetikzlibrary{shapes, shadows}
\usetikzlibrary{quotes,angles}
\pgfdeclarelayer{bg}    % declare background layer
\pgfsetlayers{bg,main}  % set the order of the layers (main is the standard layer)
\usetikzlibrary{shapes.geometric,calc}
\usepgfplotslibrary{fillbetween}
\pgfplotsset{
	%every tick label/.append style={scale=0.5},
	every axis label/.append style={font=\small},
	compat=newest,
}
\tikzset{every picture/.style={font=\small}}
% -------------------------------------- Електричні кола ------------------------------------------------
\usetikzlibrary{circuits.ee.IEC}
\tikzset{circuit ee IEC,
every info/.style=red,
semithick,
every info/.style={font=\footnotesize},
small circuit symbols,
circuit declare symbol = ampermeter,
circuit declare symbol = voltmeter,
circuit declare symbol = galvanometer,
set ampermeter graphic = {draw,generic circle IEC, minimum size=5mm,info=center:A},%
set voltmeter graphic = {draw,generic circle IEC, minimum size=5mm,info=center:V},
set galvanometer graphic = {draw,generic circle IEC, minimum size=5mm,info=center:G},
}
% -------------------------------------- Паттерни -------------------------------------------------------
\usetikzlibrary{patterns}
\tikzstyle{ground}=[fill,pattern=north east lines,draw=none,minimum width=0.3,minimum height=0.6]

\begin{document}
	\begin{tikzpicture}[
		declare function = {
		L = 2e-0;
		C = 1.2e-3;
		R = 1e3;
		RI = 50;
		RL = 0.8;
		Ri = 5 ;
		E = 10;
		omegares = 1/sqrt(L*C);
		a(\x) = 1/R + RL/(RL^2 + \x^2*L^2);
		b(\x) = \x*C - \x*L/(RL^2 + \x^2*L^2);
		c(\x) = (RI + Ri) + a(\x)/(a(\x)^2 + b(\x)^2);
		d(\x) =  b(\x)/(a(\x)^2 + b(\x)^2);
		g(\x) = c(\x)^2 + d(\x)^2;
		I(\x) = E/sqrt(g(\x));
		Imin = I(omegares);
		U(\x) = E*sqrt( (1- (Ri+ RI)*c(\x)/g(\x))^2 + ((Ri+ RI)*d(\x)/g(\x))^2 );
		Umax = U(omegares);
		},
]
		\begin{axis}[clip = false,
				% === Налаштування сітки ===
% === Налаштування сітки ===
				grid = both,
				grid style={line width=.1pt, draw=gray!10},
				major grid style={line width=.2pt,draw=gray!50},
				minor tick num = 4,
				minor grid style = {line width=.1pt,draw=gray!10},
				% === Налаштування положення координатних осей ===
				axis lines = middle,
				axis line style={-stealth},
				% === Підпис координатних осей ===
				%				xticklabels={},
				xlabel={$\omega$},
%				ylabel={$\frac{U_{0_{RLC}}}{U_{0_{\max}}}$},
				% === Положення підпису координатних осей ===
				ylabel style={left},
				xlabel style={below},
				extra tick style={% changes for all extra ticks
						tick align=outside,
						major grid style={dashed,draw=black}
					},
			    extra x tick style={
			      major tick style={
			      },
			      tick label style={
			        /pgf/number format/.cd, fixed, fixed zerofill,
			        precision=2,
			      }
			    },
				xtick style={draw=none},
				ytick style={draw=none},
				xticklabels={},
				yticklabels={},			
				extra x ticks={omegares},
				extra x tick labels={$\omega_0$},
%				extra y ticks={(E - I(0)*(Ri + RI))/Umax},
%				extra y tick labels={$\frac{\mathcal{E} - I(0)\cdot R_i}{U_{\max}}$},
				% === Вибір підписів шкали для відображення ===
				xtick = {},
				ytick = {},
				% === Налаштування мінімальних та максимальних значень координат ===
				xmin = 0*omegares,
				xmax =  3*omegares,
				ymin = {(E - I(0)*(Ri + RI))/Umax},
				ymax =  1,
				% === Налаштування розміру графіка ===
				width=1\linewidth,
%				height=0.95\textheight,
			]

			\addplot [ultra thick,samples = 500, green!50!black, thick, domain=0*omegares:3*omegares, blue] {U(x)/Umax};
			\addplot [ultra thick,samples = 500, green!50!black, thick, domain=0*omegares:3*omegares, red] {I(x)/I(0)};
			\legend{$\frac{U_{0_{RLC}}}{U_{0_{\max}}}$, $\frac{I_0(\omega)}{I_0(\omega \to 0)}$}	
		\end{axis}
	\end{tikzpicture}
\end{document}



%\begin{wrapfigure}[15]{L}{0.25\linewidth}\centering
%	\begin{tikzpicture}[framed,background rectangle/.style={draw=gray!50, fill=gray!5}]
%		\pgfmathsetmacro{\IC}{4.5}
%		\pgfmathsetmacro{\IL}{3.5}
%		\pgfmathsetmacro{\PhL}{-70}
%		\pgfmathsetmacro{\IR}{\IL*cos(\PhL)}
%		\pgfmathsetmacro{\DCL}{\IC + \IL*sin(\PhL)}
%		\pgfmathsetmacro{\PhI}{atan(\DCL/\IR)}
%		\draw[-latex, cyan] (0,0) -- ++(2,0) node[right, color=black] {$\vec{U}_{0}$} ;
%		\draw[-latex,  green!50!black, ultra thick] (0,0) -- ++({\IL*cos(\PhL)},0) node[below right, color=black] {$\vec{I}_{0_{R_L}}$};
%		\draw[-latex,  red, ultra thick] (0,0) -- ++(0,\IC) node[left, color=black] {$\vec{I}_{0_C}$}  coordinate (IC);
%		\draw[-latex,  blue, ultra thick] (0,0) -- ++(\PhL:\IL) node[left, color=black] {$\vec{I}_{0_L}$} coordinate (IL);
%		\coordinate (I) at (\IR,\DCL) ;
%		\draw[-latex] (0,0) -- (I) node[sloped, right, color=black, ultra thick] {$\vec{I}_{0}$};
%		\draw[dashed] (IL) -- (I) (IC) -- (I);
%		\draw (0,0) ++(0.4,0) arc(0:\PhL:0.4) node[pos=0.8, right] {$\phi_L$};
%		\draw (0,0) ++(0.4,0) arc(0:\PhI:0.4) node[pos=0.9, right] {$\phi$};
%	\end{tikzpicture}
%	\caption{Векторна діаграма струмів}
%	\label{pic:paralell_vect_diagramm}
%\end{wrapfigure}