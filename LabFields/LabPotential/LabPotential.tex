% !TeX program = lualatex
% !TeX encoding = utf8
% !TeX spellcheck = uk_UA
% !BIB program = biber
% !TeX root =../LabWork.tex

%\addbibresource{LabWork1.bib}
\expandafter\graphicspath\expandafter{\expandafter{\currfilebase/pic}}
\usetikzlibrary{arrows.meta}
\tikzset{
every info/.style={font=\small},
}
\newcommand\Ground{%
\mathbin{\text{\begin{tikzpicture}[circuit ee IEC]
\draw (0,1ex) to (0,0) node[ground,rotate=-90,xshift=.3ex] {};
\end{tikzpicture}}}%
}

\keywords{Закон Кулона, електричне поле, електричний заряд}
\abstract{%
Визначити напруженість та потенціал заряджених тіл різної форми та різних взаємних розташувань.
}
%\apparatus{штатив, крутильний динамометр, джерело високовольтної напруги, набір металевих кульок, металева пластина.}
\chapter{Електричні поля та потенціали заряджених тіл}
\makeworktitle

\section*{Рекомендована література }
\begin{enumerate}
\item \fullcite{BK79}\\[0.5ex]
    \emph{Книга знайомить з історичними подіями, пов'язаними з відкриттям  закону Кулона, допомагає краще зрозуміти багато проблем, які виникали у вчених щодо його експериментальної перевірки.}\\[0.5ex]
\end{enumerate}


%======================================================================================================================%

\section{Теоретичні відомості}




\section{Ідея експериментів та устаткування}




\section{Хід експерименту}



\section*{Контрольні питання}



\section*{Розрахункове завдання}




