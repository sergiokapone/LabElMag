% !TeX program = lualatex
% !TeX encoding = utf8
% !TeX spellcheck = uk_UA
% !BIB program = biber
% !TeX root =../LabWork.tex

%\addbibresource{LabWork1.bib}
\expandafter\graphicspath\expandafter{\expandafter{\currfilebase/pic}}
\usetikzlibrary{arrows.meta}
\tikzset{
every info/.style={font=\small},
}

\abstract{Експериментально визначити горизонтальну і вертикальну складові, а також магнітне нахилення місцевого геомагнітного поля.}

\keywords{}

%======================================================================================================================%

\chapter{Вимірювання магнітного поля Землі}
\makeworktitle


\section{Теоретичне підґрунтя}
\subsection{Земний магнетизм}

Про існування магнетизму було відомо з глибокої давнини. Вважається, що перший компас з'явився в Китаї. У 1600 році в книзі  <<Про магніті, магнітних тілах і про великий магніт~--- Землю>> У.~Гільбертом  було дано уявлення про причини земного магнетизму. У 1785 почалися розробки способу вимірювання напруженості магнітного поля, що базується на методі крутного моменту, запропонованому Ш. Кулоном. У 1839 К. Гаусс теоретично обґрунтував метод вимірювання горизонтальної складової вектора магнітного поля планети. На початку XX ст. було визначено зв'язок між магнітним полем Землі і її будовою. 

Однак, факт існування цього загадкового фізичного явища природи для сучасної науки як і раніше багато в чому неясний. Вже з середини XX століття загальноприйнято, що походження геомагнітного поля і основні чинники його еволюції пов'язані з процесами в рідкому зовнішньому ядрі Землі. Причина подібної впевненості полягає в наступному. Сукупність спостережних даних про магнітне поле Землі переконує нас в тому, що воно має планетарний характер і його джерела повинні знаходитися глибоко під поверхнею Землі. 

Спроба пов'язати таке магнітне поле з величезним постійним магнітом входить в протиріччя з помітним зростанням температури вглиб Землі. Справді, феромагнітні властивості зникають при досягненні критичної температури, яка називається \emph{точкою Кюрі}, та й сам образ гігантського постійного магніту десь в глибині Землі не виглядає реалістичним.

Ще одним джерелом магнітного поля Землі в принципі міг би служити розподіл зарядів, в результаті якого область між земною поверхнею і іоносферою являє собою гігантський конденсатор. Обертання Землі призводить до руху заряду цього конденсатора і електричний струм, який виникає таким чином, створює магнітне поле. Однак оцінки показують, що його напруженість набагато нижча ніж та, що спостерігається.

Ще одна з причин походження магнітного поля Землі може бути пов'язана з явищем електромагнітної індукції Фарадея. Цей механізм генерації магнітного поля називається \emph{механізмом динамо}. Вперше механізм динамо був запропонований на початку XX століття  Дж. Лармором для пояснення походження магнітного поля Сонця. Пізніше з дією цього механізму стали пов'язувати походження магнітних полів майже всіх небесних тіл, що мають магнітне поле. Суть механізму динамо зазвичай пояснюють як створення магнітного поля завдяки руху електропровідної рідини (плазми). Однак, пояснити походження геомагнітного поля механізмом динамо теж непросто. Справа в відомому правилі Ленца, згідно з яким додатковий струм, який з'являється в рамці зі струмом, що рухається в початковому (затравочному) магнітному полі, напрямлений таким чином, щоб зменшити затравочне магнітне поле. Іншими словами, обертання потоків провідної рідини не може призводити до самозбудження магнітного поля. 

Вказати конкретний реалістичний механізм, який призводить до підсилення затравочного магнітного поля в результаті дії електромагнітної індукції, вдалося лише в 50-60 рр. XX ст. Ганнесом Альвеном. Суть ідеї полягає в тому, що магнітне поле у вакуумі, яке створюється електричним струмом, перпендикулярно до цього струму. Однак виявилося, що магнітне поле в хаотичному, турбулентному або конвективному потоці, усереднене по пульсаціям цього потоку, набуває компоненти, паралельної до електричного струму. Якщо спочатку у нас заряджена рідина, що диференціально обертається (різні частини повертаються навколо загальної осі обертання з різною кутовою швидкістю) створює <<звичайне>> магнітне поле перпендикулярне до струму, то згодом рідина буде захоплювати магнітне поле і витягувати його вздовж напряму свого руху, утворюючи тороїдальні силові лінії. Цей механізм створення тороїдального поля відомий як омега-ефект.

\section{Хід роботи}



\section{Завдання}


\section{Контрольні запитання}



\section{Розрахункові завдання}




\section{Результи вимірювань та обробка експериментальних даних}


\section{Обговорення результатів}


\section*{Висновки}

\end{document}
